\chapter{Introduction REVISARLA}

The concept of measurement in quantum mechanics has been a fuzzy concept ever since the origins of the quantum theory because there is not a precise description of what happens when physicists say that the wave function \say{collapses}. This lack of description of the moment of the collapse and the inability to observe this phenomenom directly lead to different interpretations of quantum mechanics (CITAR Decoherence, the measurement problem, and interpretations of quantummechanics- SCHLOSSHAUER). In this work we do not intent to solve this problem in the quantum mechanics frame but we will show an approach to study this problem by looking at a classical analogue of the spin and describing in classical terms the effect that an environment has in the outcome of what we will call a measurement. \par

When considering the effects of an environment enclosing a system of interest, most of the times you arrive to a problem that is difficult model and solve. Generally when trying to model an environment it consists of a large number of degrees of freedom, this leads to adopting tools such as Langevin dynamics and statistical methods to try to solve this problem, this makes microscopic effects on the system have a role in the macroscopic dynamics.\par 

Some of the first approaches to model environments in statistical physics where made to model properly the Browinian motion (CITE CITAR R. Brow, Philosophical Magazine N. S., 4, 161, 1828 A. Einstein, Ann. der Phys., 17, 549, 1905 R. Kubo, Science, 233, 330, 1986.). The original Langevin equation (CITAR CITE Langevin, P. (1908). "Sur la théorie du mouvement brownien ) was used to describe Brownian motion, this approach consists on describing the system in a way that all of the effects of the environment are contained in a time-dependent force term and a velocity dependent friction of damping term. The force term behaves as a stochastic process representing the \say{randomness} of the environment affecting the central system. This method is widely used to study the properties of the central system without going into the details of the environment. It is important to state here that the environment is and will be  considered as a heat bath in the thermodynamic sense, where the central system can exchange energy back and forth in time until thermal equilibrium is reached.\par 


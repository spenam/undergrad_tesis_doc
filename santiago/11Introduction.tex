\chapter{Introduction}

The concept of measurement in quantum mechanics has been a fuzzy concept ever since the origins of the quantum theory because there is not a precise description of what happens when physicists say that the wave function \say{collapses}. This lack of description about the moment of the collapse and the inability to observe this phenomenon directly lead to different interpretations of quantum mechanics \cite{schlosshauer2005decoherence}. In this work we do not intent to solve this problem in the quantum mechanics frame but instead we will show an approach to study this problem by looking at a classical analogue of the spin and will describe in classical terms the effect that an environment has in the outcome of what we will call a measurement. \par

When considering the effects of an environment enclosing a system of interest, most of the times you arrive to a problem that is difficult to model and solve. Generally when trying to model an environment it consists of a large number of degrees of freedom, this leads to adopting tools such as Langevin dynamics and statistical methods to try to solve this problem, this makes microscopic effects on the system have a role in the macroscopic dynamics.\par 

Some of the first approaches to model environments in statistical physics where made to model properly the Brownian motion \cite{brown1828philosophical}\cite{einstein1905motion}\cite{kubo1986brownian}. The original Langevin equation \cite{langevin1908theorie} was used to describe this type of motion, this approach consists on describing the system in a way that all of the effects of the environment are contained in a time-dependent force term and a velocity dependent friction of damping term. The force term behaves as a stochastic process representing the \say{randomness} of the environment affecting the central system. This method is widely used to study the properties of the central system without going into the details of the environment. It is important to state here that the environment is and will be  considered as a heat bath in the thermodynamic sense, where the central system can exchange energy back and forth in time until thermal equilibrium is reached.\par 

The study of open systems is a very active branch in quantum and classical mechanics, most of these studies involve the use of numerical schemes to solve differential equations that may not be integrable analytically by modeling the heat bath as an ensemble of harmonic oscillators \cite{flambaum1997statistical}\cite{flambaum1997distribution}\cite{mazur1960poincare}.  Langevin dynamics on these types of model in the case of infinite number of linear harmonic oscillators in the heat bath lead to irreversible energy flow from the test particle into the bath and therefore achieves thermalization when the test particle posses a Boltzmann distribution. Several of the studies made before regarding modeling of open systems as finite heat baths coupled to a test particle also managed thermalization of the test particle \cite{flambaum1997statistical}\cite{ford1998radiating}\cite{smith2008thermalization}\cite{hasegawa2011classical}. This type of modeling is closely related to the one used in dissipation involved in quantum mechanics where the system of interest is coupled to a set of non interacting harmonic oscillators that posses a continuous linear frequency distribution known as the Caldeira-Leggett model \cite{caldeira1983quantum}. \par

In this work we will use the model described by Leggett et al. \cite{leggett1987dynamics} for dissipative two state quantum systems and treat it as a classical problem described of a double well potential, we will couple this system to a finite heat bath and study its dynamics by steadily increasing the number of harmonic oscillators in the heat bath. This type of modeling will give some insight about a classical analogue of the quantum spin measurement and will help to describe some of the effects that the environment (in this case the heat bath) will have in the result of the final dynamics of a dissipative system that can be considered as a classical measurement analogue of the quantum one. We will show the analytical properties of the particle in the double well potential and how it is a chaotic system when started initially in the unstable equilibrium point. Using numerical symplectic algorithms we will solve the equations of motions for different numbers of harmonic oscillators in the heat bath until we achieve the behavior expected of dissipative dynamics of an infinite heat bath but using a finite number of modes.

We can mimic randomness of the quantum measurement in this system by using the chaotical properties of this classical scenario. As the unstable equilibrium point is chaotic as its dynamics evolves in time, when we couple this central system to a finite heat bath on an initial \say{neutral} state, due to the heat bath not being perfectly symmetric the particle in the central system will amplify this asymmetries and will lead to the particle falling on different sides of the double well potential. In this picture we can compare it to a certain degree to what happens on a measurement of a two state quantum system, say the spin measurement of the Stern-Gerlach experiment.


\chapter{Conclusion}
This work showed how a double well system coupled to a finite heat bath of harmonic oscillators shows dissipative behavior for the double well system without the need of the bath to be infinite. We showed that even for a number around $\sim 15$ harmonic oscillators in the heat bath the system already shows this type of dynamics. It is an interesting result to see that the coefficient $s$ in the frequency spectra for the quantum case does not really have a role in the classical version of the model. We see the typical behavior that is expected in the infinite numbers of oscillators when the initial energy of the heat bath, the potential barrier of the double well and the coupling constant is increased or decreased meaning that this approximation of the model is viable to reproduce these types of dynamics to explore other systems.\par 

Regarding the the stable behavior of the particle on one of the wells we can picture it as a stable classical measurement (the particle when observed can be found either on the right or the left well), as we have said before, this behavior resembles a classical analogue of a measurement of a quantum two state system. In this approach to the model we have total control of the initial state of the heat bath, this can be exploited into having full knowledge on how the outcome of the stable behavior of the central system. We can manipulate the initial conditions of the heat bath at will to always produce the outcome desired. In this way the initial randomness of the unbiased unstable equilibrium condition on the central system is no longer random because the assymetries of the initial conditions of the heat bath are amplified affects the real measurement outcome.\par 

The quantum version of this model is currently being studied to understand more the randomness involved in the process of a quantum measurement, this project serves as a starting point on a possibility of not seeing a quantum measurement as a random process intrinsically but as a randomness induced by the uncertainty of the initial assymetries of the heat bath coupled to the quantum system when the measurement takes place.